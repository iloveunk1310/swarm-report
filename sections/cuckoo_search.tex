% sections/cuckoo_search.tex – Cuckoo Search section

\section{Cuckoo Search}
\subsection{Giới thiệu thuật toán}
Cuckoo Search (CS) là một thuật toán metaheuristic được đề xuất bởi Xin-She Yang và Suash Deb vào năm 2009,\cite{yang2009cuckoo,yang2014cuckoo} lấy cảm hứng từ hành vi sinh sản ký sinh của loài chim cu (cuckoo), kết hợp với cơ chế Lévy flight để khám phá không gian tìm kiếm hiệu quả.\cite{li2022levy}

Cuckoo Search thuộc nhóm \textit{Swarm Intelligence} tương tự như PSO hay ABC, với ý tưởng là mỗi cá thể (tổ chim) đại diện cho một nghiệm ứng viên trong không gian tìm kiếm. Quá trình tiến hóa xảy ra thông qua việc các tổ chim được thay thế dần bằng các nghiệm tốt hơn.

\subsection{Ý tưởng thuật toán}
Mỗi con chim cu đẻ trứng vào tổ của các loài chim khác. Nếu trứng bị phát hiện là "kẻ lạ", chủ tổ sẽ loại bỏ trứng đó hoặc bỏ tổ và xây tổ mới. Cuckoo Search mô phỏng hành vi này bằng cách:
\begin{itemize}
    \item Mỗi "tổ" đại diện cho một nghiệm trong không gian tìm kiếm.
    \item Một số trứng (nghiệm) mới được tạo ra thông qua Lévy flight từ các tổ hiện tại.
    \item Một tỷ lệ $p_a$ các tổ xấu nhất sẽ bị loại bỏ và thay bằng các tổ mới ngẫu nhiên.
    \item Tổ tốt nhất được giữ lại qua các thế hệ.
\end{itemize}

\subsection{Cơ sở toán học của thuật toán}
Giả sử không gian tìm kiếm có $d$ chiều và $n$ tổ chim. Mỗi tổ $x_i = [x_{i1}, x_{i2}, \ldots, x_{id}]$ tương ứng với một nghiệm.  
Tại mỗi vòng lặp, tổ mới được tạo ra theo công thức:
\begin{equation}
    x_i^{(t+1)} = x_i^{(t)} + \alpha \cdot \text{Levy}(\beta)
\end{equation}
Trong đó:
\begin{itemize}
    \item $\alpha$ là hệ số bước bay (step size),
    \item $\text{Levy}(\beta)$ là bước nhảy theo phân phối Lévy với tham số $\beta$ (thường $\beta = 1.5$).\cite{li2022levy}
\end{itemize}

Phân phối Lévy thường được sinh ra theo công thức Mantegna:\cite{mantegna1994fast}
\begin{equation}
    s = \frac{u}{|v|^{1/\beta}}, \quad
    u \sim N(0, \sigma_u^2), \quad
    v \sim N(0, 1)
\end{equation}
với
\begin{equation}
    \sigma_u = \left[ \frac{\Gamma(1+\beta)\sin(\pi \beta/2)}
    {\Gamma((1+\beta)/2) \beta 2^{(\beta-1)/2}} \right]^{1/\beta}
\end{equation}

Sau mỗi lần cập nhật, các tổ bị phát hiện (với xác suất $p_a$) sẽ bị thay thế ngẫu nhiên:
\begin{equation}
    x_i^{(t+1)} = x_i^{(t)} + r \cdot (x_j^{(t)} - x_k^{(t)}),
\end{equation}
trong đó $x_j, x_k$ là hai tổ ngẫu nhiên khác nhau, và $r$ là một số ngẫu nhiên trong $[0,1]$.

\subsection{Triển khai thuật toán}
\subsubsection{Cách hoạt động của thuật toán}
\begin{enumerate}
    \item \textbf{Khởi tạo quần thể:} Tạo $n$ tổ ngẫu nhiên trong không gian tìm kiếm.
    \item \textbf{Đánh giá:} Tính giá trị hàm mục tiêu (fitness) của từng tổ.
    \item \textbf{Bay Lévy:} Tạo các tổ mới bằng cách bay Lévy từ các tổ hiện tại.
    \item \textbf{Cập nhật:} Nếu tổ mới tốt hơn, thay thế tổ cũ.
    \item \textbf{Phát hiện trứng lạ:} Với xác suất $p_a$, thay thế các tổ tệ nhất bằng tổ mới ngẫu nhiên.
    \item \textbf{Ghi nhận nghiệm tốt nhất:} Cập nhật tổ có giá trị hàm mục tiêu nhỏ nhất.
    \item \textbf{Lặp lại} cho đến khi đạt số vòng lặp tối đa hoặc hội tụ.
\end{enumerate}

\subsubsection{Mã giả }
\begin{algorithm}[H]
\caption{Thuật toán Cuckoo Search}
\begin{algorithmic}[1]
\STATE Khởi tạo $n$ tổ $x_i$ ngẫu nhiên trong không gian tìm kiếm
\STATE Đánh giá giá trị hàm mục tiêu $f(x_i)$ cho mỗi tổ
\FOR{$t = 1$ đến $T_{max}$}
    \STATE \textbf{(Pha 1: Lévy flight)}\\
    \FOR{mỗi tổ $x_i$}
        \STATE Sinh bước Lévy $s \sim \text{Levy}(\beta)$
        \STATE $x_i' = x_i + \alpha \cdot s$
        \STATE Nếu $f(x_i') < f(x_i)$ thì cập nhật $x_i = x_i'$
    \ENDFOR
    \STATE \textbf{(Pha 2: Phát hiện trứng)}\\
    \STATE Chọn ngẫu nhiên một phần $p_a$ các tổ tệ nhất
    \STATE Thay thế chúng bằng tổ mới ngẫu nhiên trong không gian tìm kiếm
    \STATE Cập nhật tổ tốt nhất hiện tại
\ENDFOR
\STATE Xuất ra nghiệm tốt nhất tìm được
\end{algorithmic}
\end{algorithm}
